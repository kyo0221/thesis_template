%!TEX root = ../thesis.tex
\chapter*{概要}
\thispagestyle{empty}
%
\begin{center}
  \scalebox{1.5}{AIFormulaにおける経路追従ソフトウェアの検討}\\
\end{center}
\vspace{1.0zh}
%

%
本論文では自律移動ロボットにおける経路追従ソフトウェアに関する検討を行った結果について述べる.
本研究室では, AIFormulaという技術チャレンジに参加しており自律移動ロボットの研究が盛んにおこなわれている.
AIFormulaではハードウェアの変更もある程度許容されているが, 現時点の開発はソフトウェアが中心となる.
ハードウェアは経路追従するために必要なパーツがすべて揃っているが, ソフトウェアはデバイスを駆動するサンプルプログラムが用意されているのみである.
競技という性質上, 経路追従などのソフトウェアは各チームで開発することが必要となる.
そのため, 経路追従をおこなうパッケージを開発して評価をすることで, 作成した経路追従パッケージの有効性を実環境で検証する.
%
%
%
%
%
%
%
%
%
%
%
%
%

\vspace{2.0zh}

キーワード: AIFormula, 経路追従, ソフトウェア開発
%

\newpage
%%
\chapter*{abstract}
\thispagestyle{empty}
%
\begin{center}
  \scalebox{1.3}{Examination of route-following software in AIFormula}
\end{center}
\vspace{1.0zh}
%
This paper describes the results of a study on path-following software for autonomous mobile robots.
Our laboratory is participating in a technology challenge called AIFormula, in which research on autonomous mobile robots is being actively conducted.
Although AIFormula allows some modification of hardware, the current development is mainly focused on software.
While the hardware has all the necessary parts for path-following, the software only includes a sample program to drive the device.
Due to the nature of the competition, it is necessary for each team to develop its own software for path-following and so on.
Therefore, we will develop and evaluate a path-following package to verify the effectiveness of the created path-following package in a real environment.
%
\vspace{2.0zh}

keywords: AIFormula, route-following, software development
